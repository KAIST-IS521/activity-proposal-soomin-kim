\documentclass[a4paper, 11pt]{article}

\usepackage{kotex} % Comment this out if you are not using Hangul
\usepackage{fullpage}
\usepackage{hyperref}
\usepackage{amsthm}
\usepackage[numbers,sort&compress]{natbib}

\theoremstyle{definition}
\newtheorem{exercise}{Exercise}

\begin{document}
%%% Header starts
\noindent{\large\textbf{IS-521 Activity Proposal}\hfill
                \textbf{김수민}} \\
         {\phantom{} \hfill \textbf{soomin-kim}} \\
         {\phantom{} \hfill Due Date: April 15, 2017} \\
%%% Header ends

\section{Activity Overview}

IS-521 Activity Proposal로 Key Logger를 제작하는 Activity를 제안합니다. 어떤
시스템에 침투한 후에 그 시스템의 사용자로부터 정보를 얻어오는 방법 중 하나인
Key Logger를 제작해본다면, 학생들이 Key Logger의 원리에 대해 알 수 있을 것
입니다.

학생들은 이번 Activity로부터 Keyboard Device를 hooking하여 사용자의 key 입력을
받아 저장하는 간단한 Key Logger를 제작하게 됩니다.

\section{Exercises}

\begin{exercise}

  첫 번째 Exercise로는, Keyboard Driver를 Hooking하는 Code를 작성하게 됩니다.

\end{exercise}

\begin{exercise}

  두 번째 Exercise로는, 사용자로부터 읽은 입력을 적당한 형태로 Logging하는
  Code를 작성하게 됩니다.

\end{exercise}

\begin{exercise}

  세 번째 Exercise로는, Logged된 Data를 전송하는 Code를 작성하게 됩니다.

\end{exercise}

\section{Expected Solutions}

  학생들은 다음 repository~\cite{keylogger}와 비슷한 형태의 key logger를
  제작하게 됩니다.

\bibliography{references}
\bibliographystyle{plainnat}

\end{document}
